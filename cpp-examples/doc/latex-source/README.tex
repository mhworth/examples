\documentclass[a4paper,10pt]{article}

\usepackage{listings}


%opening
\title{README File for mhworth.com cpp-examples}
\author{Matt Hollingsworth}

\begin{document}
\tolerance=4000
\setlength{\emergencystretch}{200em}
\maketitle

\section{Introduction}
This README file attempts to introduce the user to the usefulness of this package and to provide instructions that are necessary for building and running the different examples.  The different examples are divided into categories, which shall be enumerataed in the next section.  The instructions included in this README apply to all categories; however, each category of examples has its own specific build requirements, which are explained externally to this README.  

\section{Packages}
cpp-examples is made up of multiple categories of examples--each category, in general, corresponds to a specific library.  Here are the categories:

\begin{itemize}
 \item ACE: The Adaptive Communications Framework \\ URL: http://www.cs.wustl.edu/~schmidt/ACE.html
 \item boost: The Boost C++ framework \\ URL: http://www.boost.org
 \item bullet: The Bullet physics engine \\ URL: http://www.bulletphysics.com/Bullet/
 \item corba: CORBA examples based on ACE+TAO \\ URL: http://www.cs.wustl.edu/~schmidt/ACE.html
 \item ogre: The OGRE 3d framework \\ URL: http://ogre3d.org/
 \item scons: A build utility that is compatible with C++ and Java \\ URL: http://www.scons.org
 \item xdaq: A C++-based framework for distributed data acquisition \\ URL: http://xdaqwiki.cern.ch
\end{itemize}


\section{Preparing to build}

To build the targets, you must download and install scons (http://www.scons.org).  scons is a wonderful python-based build utility that runs circles around make and friends, in my opinion.

If you are using eclipse, you'll have to do a tiny bit of hacking to get it to work right.  Most of the work is part
of the project itself already; all you really have to do is set the proper command to run scons by setting the eclipse
workspace variable ``scons\_cmd.''  You may do this by going to project properties $>$ C++ build $>$ and adding scons\_cmd to the list.  On windows, you will need to look in the bin directory for scons.cpp, which is a simple wrapper to execute scons.bat from eclipse.  This is necessary because eclipse will only execute executables.  scons.cpp is standalone, so all you should have to do is run

\begin{verbatim}
> CL scons.cpp
with Visual C++ or
> g++ scons.cpp
\end{verbatim}


\noindent for MinGW and friends to get the executable.  You then need to put scons.exe in your PATH somewhere.  To do this, right click My Computer, go to properties -> Advanced, and look for an Environment Variables button.  Here, you would need to append \verb|";<your scons directory>"| to the existing PATH variable under ``User Environment Variables``.

After this is done, your scons\_cmd eclipse variable will look something like

\begin{verbatim}scons C:\Python25\scons.bat <arguments>\end{verbatim}

The \verb|C:\Python25\scons.bat| is the location of your scons.bat file so that scons.exe can execute it.  All other arguments to scons.exe are forwarded to scons.

Afterward, you are ready to build whatever you wish.  Make sure you read the package-specific build instructions for dependency information, necessary environment variables, etc. If you are using eclipse, you can open up the Make Targets window and simply right click each of the premade make targets for the particular target that you wish to build.  Otherwise, you'll simply need to cd to the appropriate directory and run

\begin{verbatim}> scons <target name>\end{verbatim}

\noindent to build the proper target.


Any questions should be forwarded to Matt Hollingsworth at hollings@cern.ch.

\end{document}
